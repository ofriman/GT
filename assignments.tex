\documentclass[a4paper, 11pt, oneside]{article}

 \usepackage[margin=1in]{geometry} 
\usepackage{amsmath,amsthm,amssymb, float, enumitem}
 
\newtheorem{innercustomthm}{Theorem}
\newenvironment{theorem}[1]
  {\renewcommand\theinnercustomthm{#1}\innercustomthm}
  {\endinnercustomthm}
  
  \newtheorem{innercustomprob}{Problem}
\newenvironment{problem}[1]
  {\renewcommand\theinnercustomprob{#1}\innercustomprob}
  {\endinnercustomprob}
  
  \newtheorem{innercustomdef}{Definition}
\newenvironment{definition}[1]
  {\renewcommand\theinnercustomdef{#1}\innercustomdef}
  {\endinnercustomdef}
  
\newcommand{\N}{\mathbb{N}}
\newcommand{\Z}{\mathbb{Z}}
\newcommand\abs[1]{\left|#1\right|}
\newcommand\imark[2]{\text{i-Mark}{(\{#1\},\{#2\})}}
\DeclareMathOperator\mex{mex}
\DeclareMathOperator\opt{opt}

\begin{document}

\title{Advanced topics in graph theory Assignments}
\author{Oren Friman 301677613}
\maketitle

\section{Total unimodularity and incidence matrices}

A square matrix $M$ is said to be unimodular if det $M\in \{-1, 1\}$. A matrix $M$ (not necessarily
square) is said to be totally unimodular if each square submatrix $N$ of $M$ satisfies det $N \in \{-1, 0, 1\}$.

\begin{problem}{1.1}\label{problem1.1}
Let $G$ be an undirected graph and let $\sigma$ be an orientation of its edges. Prove that
$\mathcal{Q}(G^\sigma)$ is totally unimodular.
\end{problem}

\begin{proof}
Let $N_{k\times k}$ be a square submatrix of $\mathcal{Q}(G^\sigma)$. We proceed by induction on $N$. For the base case let $k = 1$, the determinant of such matrix is the number itself. Suppose now that the theorem holds up to $k=n$. Here we consider cases:
\begin{enumerate}[label=(\alph*)]
\item $N$ contain at least one column or row $c_i$ containing only zero elements. The determinant of $N$ can be computed by using the Laplace expansion along $c_i$ which will give us zero.
\item $N$ contain at least one column or row $c_i$ containing only one element $e_j$ not equal zero. The determinant of $N$ can be computed by using the Laplace expansion along $c_i$ which will give us $(-1)^{i+ 1} * e_j * det(N_{i,j})$ where $N_{i,j}$ is submatrix of $N$, from the induction hypothesis $det(N_{i,j}) \in\{-1, 0, 1\}$ and $e_j \in\{-1, 1\}$ therefore the theorem holds.
\item $N$ is incidence matrix, we know that the sum of rows are equal zero therefore the rows are dependent and the determinant is equal zero.
\end{enumerate}
\end{proof}

\begin{problem}{1.2}
\hfill 
\begin{enumerate}[label=1.2.\arabic*]
\item \label{problem1.2.1} Let $n \geq 3$. Prove that
\begin{equation*}
det M(C_n) = 
	 \begin{cases}
	0, & \text{if $n$ is even}\\
	2, & \text{if $n$ is odd};
	 \end{cases}
\end{equation*}
\begin{proof}
\begin{multline*}
  det M(C_n) = 
  \underbrace {
\begin{vmatrix}
1 & 0 &  \cdots & 1 \\ 
1 & 1 &  \cdots & 0 \\
0 & 1 &  \cdots & 0 \\
\vdots & \vdots & \ddots &\vdots \\
0 & 0 & \cdots & 0 \\ 
0 & 0 & \cdots & 1 \\ 
\end{vmatrix}_{n\times n}
}_\text{we will expand along first row }
= \\
\underbrace {
\begin{vmatrix}
1 & 0 &  \cdots & 0 \\ 
1 & 1 &  \cdots & 0 \\
0 & 1 &  \cdots & 0 \\
\vdots & \vdots & \ddots &\vdots \\
0 & 0 & \cdots & 0 \\ 
0 & 0 & \cdots & 1 \\ 
\end{vmatrix}_{n - 1 \times n - 1}
}_\text{we will expand along first row }
 + (-1)^{n+1}
 \underbrace {
\begin{vmatrix}
1 & 1 &  \cdots & 0 \\ 
0 & 1 &  \cdots & 0 \\
0 & 0 &  \cdots & 0 \\
\vdots & \vdots & \ddots &\vdots \\
0 & 0 & \cdots & 1 \\ 
0 & 0 & \cdots & 1 \\ 
\end{vmatrix}_{n - 1 \times n - 1}
}_\text{we will expand along first column }
= \\ \cdots =
\begin{vmatrix}
1 & 0 \\ 
1 & 1 \\ 
\end{vmatrix}
+ (-1)^{n+1}
\begin{vmatrix}
1 & 1 \\ 
0 & 1 \\ 
\end{vmatrix}
= 1 + (-1)^{n+1} 
\end{multline*}
\end{proof}

\item Use the first part of this problem to prove that an undirected graph $G$ is bipartite if and if $M(G)$ is totally unimodular.
\begin{proof}
$G$ is bipartite $\rightarrow$ $M(G)$ is totally unimodular.\\ 
Let $G$ be bipartite graph and Let $N_{k\times k}$ be a square submatrix of $M(G)$. We proceed by induction on $N$. For the base case let $k = 1$, the determinant of such matrix is the number itself. Suppose now that the theorem holds up to $k=n$. Here we consider cases:
\begin{enumerate}[label=(\alph*)]
\item $N$ contain at least one column or row $c_i$ containing only zero elements. The determinant of $N$ can be computed by using the Laplace expansion along $c_i$ which will give us zero.
\item $N$ contain at least one column or row $c_i$ containing only one element equal 1. The determinant of $N$ can be computed by using the Laplace expansion along $c_i$ which will give us $(-1)^{i+ 1} * det(N_{i,j})$ where $N_{i,j}$ is submatrix of $N$, from the induction hypothesis $det(N_{i,j}) \in\{-1, 0, 1\}$ therefore the theorem holds.
\item $N$ is incidence matrix, we know from \cite[Theorem 10]{konig_egervary}  that the rk $N \leq n - 1$ therefore the columns are dependent and the determinant is equal zero.
\end{enumerate}
$M(G)$ is totally unimodular $\rightarrow$ $G$ is bipartite.\\ 
From K$\ddot{o}$nig Theorem, we know that a graph is bipartite if and only if it has no odd cycle. Also we saw in \ref{problem1.2.1} that  totally unimodular matrix cannot contain odd cycle therefore it most be bipartite.
\end{proof}
\end{enumerate}
\end{problem}

\begin{thebibliography}{9} 
\bibitem{konig_egervary}
Elad Aigner-Horev,
The K$\ddot{o}$nig-Egerv$\acute{a}$ry theorem.
\\\texttt{http://www.elad-horev.org/Expositions/Konig\_Egervary.pdf}
\end{thebibliography}

\end{document}
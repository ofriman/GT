\documentclass[a4paper, 11pt, oneside]{article}

 \usepackage[margin=1in]{geometry} 
\usepackage{amsmath,amsthm,amssymb, float,centernot}
\usepackage[shortlabels]{enumitem}
 \usepackage[hidelinks]{hyperref}
\usepackage{xcolor}
\hypersetup{
    colorlinks,
    linkcolor={red!50!black},
    citecolor={blue!50!black},
    urlcolor={blue!80!black}
}

\let\oldquote\quote
\let\endoldquote\endquote
\renewenvironment{quote}[2][]
  {\if\relax\detokenize{#1}\relax
     \def\quoteauthor{#2}%
   \else
     \def\quoteauthor{#2~---~#1}%
   \fi
   \oldquote}
  {\par\nobreak\smallskip\hfill(\quoteauthor)%
   \endoldquote\addvspace{\bigskipamount}}

\newtheorem{innercustomthm}{Theorem}
\newenvironment{theorem}[1]
  {\renewcommand\theinnercustomthm{#1}\innercustomthm}
  {\endinnercustomthm}
  
  \newtheorem{innercustomprob}{Problem}
\newenvironment{problem}[1]
  {\renewcommand\theinnercustomprob{#1}\innercustomprob}
  {\endinnercustomprob}
  
  \newtheorem{innercustomdef}{Definition}
\newenvironment{definition}[1]
  {\renewcommand\theinnercustomdef{#1}\innercustomdef}
  {\endinnercustomdef}
  
\newcommand{\N}{\mathbb{N}}
\newcommand{\Z}{\mathbb{Z}}
\newcommand\abs[1]{\left|#1\right|}
\newcommand\imark[2]{\text{i-Mark}{(\{#1\},\{#2\})}}
\DeclareMathOperator\mex{mex}
\DeclareMathOperator\opt{opt}

\begin{document}

\title{Final take-home exam}
\author{Oren Friman 301677613}
\maketitle
				   
\begin{problem}{1}\label{problem1}
Show how to find the two non-zero eigenvalues of $K_{1,n}$ using the functional
approach. A natural alternative approach would be to memorise or dig up the spectrum of $K_{n,m}$
considered here; and then derive the answer from that. This will not be accepted as an answer.
\end{problem}

\begin{proof}
We will use \cite[Equation 1]{adjacency_matrix} to find  the two non-zero eigenvalues.
\begin{equation*}
\lambda x(u) = \sum_{v \sim u} x(v)
\end{equation*}
Now we must take $v_1$ otherwise the sum of the neighbors will be equal zero.
\begin{equation*}
\lambda x(v_1) = \sum_{v \sim v_1} x(v)
\end{equation*}
We will consider the simplest eigenvector  and we take $x_2 = \ldots = x_{n+1} = 1$.
\begin{equation*}
\lambda x_1 = n
\end{equation*}
having $\lambda = x_1 = \sqrt{n}$ gives us two eigenvalues $\pm \sqrt{n}$
\end{proof}		

\begin{problem}{2}\label{problem2}
Let $G$ be a graph. Is it true that $\lambda_n(G) < 0$ always hold? Prove your claims.
\end{problem}

\begin{proof}
No, this is not true, we know that $\sum_i \lambda_i = 0$ so it will be true only if $\lambda_1(G) > 0$.
Contradiction by example, let $A(G) = 
\begin{pmatrix}
0 & 0  \\
0& 0  \\
\end{pmatrix}$

\begin{equation*}
det(A(G) - \lambda I) =
\begin{vmatrix}
-\lambda & 0            \\ 
0             & -\lambda \\
\end{vmatrix} = (-\lambda)^2
\end{equation*}
\end{proof}		
   
\begin{problem}{3}\label{problem3}\hfill 
  \begin{enumerate}[(a)]
  \item Let $H$ be a connected spanning subgraph of a (connected) graph $G (\text{i.e., }V (H) =
V (G))$. Prove that $\lambda_1(H) \leq \lambda_1(G)$
  \item Let $H$ be a connected subgraph of a connected graph $G$ such that $\abs{V (H)} < \abs{V (G)}$.
Prove that $\lambda_1(H) < \lambda_1(G)$.
\end{enumerate}
\end{problem}

\begin{proof}

  \begin{enumerate}[(a)]
  \item \label{problem3.1}
  Let $H$ be a connected spanning subgraph of a (connected) graph $G (\text{i.e., }V (H) =
V (G))$.

Same argument as \cite[Seva]{eigenvalue_subgraph},
 without loss of generality, $H$ is obtained from $G$ by deleting some edges (and keeping all vertices). Let $A(G)$ and $A(H)$ denote the adjacency matrices, and let $x'$ be an eigenvector of $H$, belonging to the eigenvalue $\lambda_1(H)$, such that all coordinates of $x'$ are non-negative. We have then
\begin{equation*}
\lambda_1(H) = \frac{\langle x', A(H)x' \rangle }{\abs{\abs{x'}}^2} \leq  \frac{\langle x', A(G)x' \rangle }{\abs{\abs{x'}}^2}  \leq  \underset{x \neq 0}{sup} \frac{\langle x, A(G)x \rangle }{\abs{\abs{x}}^2}  = \lambda_1(G)
\end{equation*}
  indeed, if all coordinates of $x'$ are strictly positive, then the first inequality is strict, and if $x'$ has zero coordinates, then the second inequality is strict (by Perron-Frobenius \cite[Theorem 1]{perron_frobenius}, which says that the supremum is attained on a vector with all coordinates distinct from 0).
  
  \item 

\begin{quote}{\cite[Proposition 1.3.9]{introduction-spectra}}{For any vertex $u$ of a connected graph $G$, we have $\lambda_1(G-u) < \lambda_1(G)$.}
\hfill\break
Let $A = 
\begin{pmatrix}
A' & r  \\
r^T & 0  \\
\end{pmatrix}
$, where $A' = A(G-u)$, and let $x$ be a unit eigenvector of $A'$ corresponding to $\lambda_1(G-u)$. if $y = \begin{pmatrix}
x \\
0 \\
\end{pmatrix}$
then $y^Ty = 1$ and $\lambda_1(G - u) = y^TAy \leq \lambda_1(G)$. if equality holds then y is an eigenvector of A corresponding to $\lambda_1(G)$; but this is a contradiction because $y$ has a zero entry.
\end{quote}

\cite[Proposition 1.3.9]{introduction-spectra} together with the first part of this question \ref{problem3.1}  prove this question.
\end{enumerate}


\end{proof}		

\begin{problem}{4}\label{problem4}
Prove that
\begin{equation*}
max \Big\{ \overline{d}(G), \sqrt{\Delta(G)} \Big\} \leq \lambda_1(G) \leq \Delta(G)
\end{equation*}
where here $\overline{d}(G)$ is the average degree of G and $\Delta(G)$ is the maximum degree of G.
\end{problem}

\begin{proof}


\end{proof}		

\begin{problem}{5}\label{problem5}
\end{problem}

\begin{proof}

\end{proof}		

 

\begin{thebibliography}{9} 
\bibitem{adjacency_matrix}
Elad Aigner-Horev,
The adjacency matrix of a graph.
\\\texttt{http://www.elad-horev.org/Expositions/Adjacency\_Matrix.pdf}

\bibitem{eigenvalue_subgraph}
Spectral radius of a proper subgraph.
\\\texttt{https://mathoverflow.net/questions/69807/spectral-radius-of-a-proper-subgraph}

\bibitem{perron_frobenius}
Elad Aigner-Horev,
The Perron-Frobenius theorem for connected graphs.
\\\texttt{http://www.elad-horev.org/Expositions/Perron\_Frobenius.pdf}

\bibitem{introduction-spectra}
Dragos Cvetkovic, Peter Rowlinson,
Introduction-Spectra-Mathematical-Society-Student (Kindle Edition)
\end{thebibliography}

\end{document}
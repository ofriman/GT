\documentclass[a4paper, 11pt, oneside]{article}

 \usepackage[margin=1in]{geometry} 
\usepackage{amsmath,amsthm,amssymb, float, enumitem,centernot}
 \usepackage[hidelinks]{hyperref}
\usepackage{xcolor}
\hypersetup{
    colorlinks,
    linkcolor={red!50!black},
    citecolor={blue!50!black},
    urlcolor={blue!80!black}
}

\newtheorem{innercustomthm}{Theorem}
\newenvironment{theorem}[1]
  {\renewcommand\theinnercustomthm{#1}\innercustomthm}
  {\endinnercustomthm}
  
  \newtheorem{innercustomprob}{Problem}
\newenvironment{problem}[1]
  {\renewcommand\theinnercustomprob{#1}\innercustomprob}
  {\endinnercustomprob}
  
  \newtheorem{innercustomdef}{Definition}
\newenvironment{definition}[1]
  {\renewcommand\theinnercustomdef{#1}\innercustomdef}
  {\endinnercustomdef}
  
\newcommand{\N}{\mathbb{N}}
\newcommand{\Z}{\mathbb{Z}}
\newcommand\abs[1]{\left|#1\right|}
\newcommand\imark[2]{\text{i-Mark}{(\{#1\},\{#2\})}}
\DeclareMathOperator\mex{mex}
\DeclareMathOperator\opt{opt}

\begin{document}

\title{Final take-home exam}
\author{Oren Friman 301677613}
\maketitle
				   
\begin{problem}{1}\label{problem1}
Show how to find the two non-zero eigenvalues of $K_{1,n}$ using the functional
approach. A natural alternative approach would be to memorise or dig up the spectrum of $K_{n,m}$
considered here; and then derive the answer from that. This will not be accepted as an answer.
\end{problem}

\begin{proof}
We will use \cite[Equation 1]{adjacency_matrix} to find  the two non-zero eigenvalues.
\begin{equation*}
\lambda x(u) = \sum_{v \sim u} x(v)
\end{equation*}
Now we must take $v_1$ otherwise the sum of the neighbors will be equal zero.
\begin{equation*}
\lambda x(v_1) = \sum_{v \sim v_1} x(v)
\end{equation*}
We will consider the simplest eigenvector  and we take $x_2 = \ldots = x_{n+1} = 1$.
\begin{equation*}
\lambda x_1 = n
\end{equation*}
having $\lambda = x_1 = \sqrt{n}$ gives us two eigenvalues $\pm \sqrt{n}$

\end{proof}		
   
\begin{thebibliography}{9} 
\bibitem{adjacency_matrix}
Elad Aigner-Horev,
The adjacency matrix of a graph.
\\\texttt{http://www.elad-horev.org/Expositions/Adjacency\_Matrix.pdf}
\end{thebibliography}
\end{document}